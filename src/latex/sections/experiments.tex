\section{Experiments}
This section sketches typical experimental reporting.

\paragraph{Datasets} Specify dataset sources, splits, and licenses.

\paragraph{Baselines} Compare against strong baselines, not only weak ones.

\paragraph{Metrics} Use clear, standard metrics and report central tendency with dispersion (e.g., mean $\pm$ stdev).

\paragraph{Ablations} Vary model components to understand contributions.

\paragraph{Reproducibility} Fix seeds, log versions, and provide scripts. Hyperparameters should be listed or referenced.

An illustrative quantitative table is shown in Table~\ref{tab:results}.

\begin{table}[h]
  \centering
  \begin{tabular}{lcc}
    \toprule
    Model & Accuracy (\%) & Params (M) \\
    \midrule
    Baseline CNN~\citep{krizhevsky2012imagenet} & 75.3 & 60 \\
    Transformer~\citep{vaswani2017attention} & 82.1 & 85 \\
    BERT-base~\citep{devlin2018bert} & 84.5 & 110 \\
    GPT-3 small~\citep{brown2020language} & 86.0 & 125 \\
    \bottomrule
  \end{tabular}
  \caption{Illustrative results (fabricated for template purposes).}
  \label{tab:results}
\end{table}
